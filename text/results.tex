In the simplest case, we can use two points to determine a linear relationship
using ordinary least squares. We used a two point fit with $^{137}$Cs and
$^{241}$Am to give the model

\begin{equation}
E = 1.181 + 0.281 C
\end{equation}

We can then use this model to estimate at which bins lines from $^{133}$Ba are
expected to fall in.

\begin{table}[H]
  \begin{center}
    \begin{tabular}{l|c|c|r}
      \textbf{Source} & \textbf{Energy (keV)} & \textbf{Estimated Energy (keV)} & \textbf{Percent Error}\\
      \hline
      $^{133}$Ba    &  80.997    & 80.684   &  0.1632 \\
                    &  356.017   & 356.110  &  -0.0261 \\
    \end{tabular}
  \end{center}
\end{table}

We see that the two estimated energies are quite close to the actual energies. In addition to using two points,
it is also worth considering using additional data points, including some at higher energies. An ordinary least
squares fit was performed on several points to obtain the line below. The solid line in Figure \ref{fig:fit} shows a fit using all data points,
while the dashed line shows the two-point relationship.

\begin{figure}[H]
\label{fig:fit}
\begin{center}
\includegraphics[width=.6\linewidth]{calibration_fit.eps}
\caption{Resulting calibration fit}
\end{center}
\end{figure}
