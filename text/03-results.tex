In the simplest case, we can use two points to determine a linear relationship using ordinary least squares (OLS) regression.
We used a two point fit with \ce{$^{137}$Cs} and \ce{$^{241}$Am} to give the model

\begin{equation}
\label{eq:cal}
E_k = 1.181 + 0.281k
\end{equation}
where $k$ is the bin number.


To evaluate the performance of this model, we estimate at which bins lines from \ce{$^{133}$Ba} are expected to fall in, and compare this to what has been measured.
Table \ref{tab:barium} shows measured and estimated energies for \ce{^{133} Ba}, as well as the percent error.
We see that the two estimated energies are quite close to the actual energies.

\begin{table}
\renewcommand{\arraystretch}{1.3}
\caption{Measured and estimated source energies for \ce{^{133} Ba}}
\label{tab:barium}
\centering
\begin{tabular}{l|c|c|r}
\hline
\bfseries Source & \bfseries Energy (keV) & \bfseries Estimated Energy (keV) & \bfseries Percent Error\\
\hline\hline
      $^{133}$Ba    &  80.997    & 80.684   &  0.1632 \\
                    &  356.017   & 356.110  &  -0.0261 \\
\hline
\end{tabular}
\end{table}

In addition to the two points used here, it is also worth considering using additional data points, including some at higher energies.
OLS was performed on several points to obtain the line below.
The solid line in Figure \ref{fig:fit} shows a fit using all data points, while the dashed line shows the two-point relationship.

\begin{figure}
\centering
\includegraphics[width=3.5in]{calibration_fit.eps}
\caption{Resulting linear fit of gamma-ray energies to channel number. The solid red line shows the fit using all data points, while the dashed blue results from the two-point fit.}
\label{fig:fit}
\end{figure}

The spectra shown in \Fref{fig:spectra} could then be translated into energies using \Fref{eq:cal}.
Calibrated spectra are shown in \Fref{fig:spectra_cal}.
\begin{figure*}[t!]
\centering
\includegraphics[width=6.5in]{calibrated_spectra.eps}
\caption{Energy-calibrated spectra captured by the HPGe used in the calibration procedure. \ce{^{241}Am} (left), \ce{^{133}Ba} and \ce{^{137}Cs} (center), and \ce{^{60}Co} and \ce{^{152}Eu} (right).}
\label{fig:spectra_cal}
\end{figure*}
