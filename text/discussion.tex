The resulting fit from our linear regression can then be used for mapping
MCA channels to gamma-ray energy. For routine lab meaurements and estimations,
experts often time only use two energy measurements from a single source to determine a linear fit
\cite{gilmore_2011}. While a linear fit is generally sufficient, some
specific applications may require nonlinear fitting. This can easily be accomplished
my creating a feature matrix using higher powers of the measured centroids $C_i$ and
again fitting with ordinary least squares.

Aside from using a higher-order model, one could also improve the fit used to
determine the peaks. Here, a Gaussian model was assumed, and no linear offset
or background subtraction was performed. To more accurately determine the peak centers,
one could fit a Gaussian superimposed on a linear model. In our case, the
simplified fit provided results sufficient for the purpose of this exercise,
so only a single Gaussian peak was used.
