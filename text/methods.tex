Measurements of various gamma-ray sources using a coaxial HPGe were performed
by Dr. Ross Barnowski. The particular sources measured are given in Table
\ref{tab:src}. The lines of interest were primarily chosen based on their
branching ratios, as those with the highest branching ratios are often times
the most visible.

\begin{table}[H] \label{tab:src}
  \begin{center}
    \begin{tabular}{l|r}
      \textbf{Source} & \textbf{Energy (keV)}\\
      \hline
      $^{241}$Am    &  59.5    \\
      $^{133}$Ba    &  80.997  \\
                    &  356.017 \\
      $^{60}$Co     &  1173.2  \\
                    &  1332.5  \\
      $^{137}$Cs    &  661.6   \\
      $^{152}$Eu    &  121.8   \\
                    &  344.3   \\
                    &  1408.0  \\
    \end{tabular}
    \caption{Gamma-ray lines used in the calibration}
  \end{center}
\end{table}

For each gamma-ray energy of interest, the line was found in the spectrum for
the associated source and fitted with a Gaussian function in the form

\begin{equation}
G(x; A, \mu, \sigma) = A\exp\big(-\frac{(x-\mu)^2}{2\sigma^2}\big)
\end{equation}

The relevant parameter for the energy calibration is the centroid $\mu$ of the fit,
though it is worth noting that the other parameters also contain valuable information,
for example, in determining the full-width at half-max (FWHM) energy resolution.

After determining the centroid $C_i = \mu_i$ for each energy of interest, a
regression is performed between $C_i$ and the associated energy $E_i$. While it
may seem obvious to simply perform a linear fit, there are slight nonlinearities
in the relationship between $C_i$ and $E_i$, leading us to find a find a general
nonlinear fit between the two. In particular, we wish to find coefficients
$\vec{a} = (a_0, a_1, \dots, a_D)$ such that

\begin{equation}
    \begin{bmatrix}
        E_0 \\ E_1 \\ \vdots \\ E_n
    \end{bmatrix} =
    \begin{bmatrix}
        1 & C_0 & C_0^2 & \dots & C_0^D \\
        1 & C_1 & C_1^2 & \dots & C_1^D \\
        \vdots & \vdots & \vdots & \vdots & \vdots \\
        1 & C_n & C_n^2 & \dots & C_n^D \\
    \end{bmatrix}
    \begin{bmatrix}
        a_0 \\ a_1 \\ \vdots \\ a_D
    \end{bmatrix}
\end{equation}

where $n$ is the number of gamma-ray lines used and $D$ is the desired model
order. A model order $D = 3$ was chosen to account these small nonlinearities.
Note that a more robust nonlinear model order $D$ could be determined by using
an additional set of gamma-ray lines not used in training.
