Measurements of various gamma-ray sources using a coaxial HPGe were performed and provided by
by Dr. Ross Barnowski. The measurements were performed using a 13-bit resolution MCA,
yielding 8192-bin spectra. The procedure for taking the measurements presumably involved placing a
source at a specific location and recording counts for a period of time, and
repeating for each source. The source location should be the same for each source,
and should be close enough to yield a high number of counts in a reasonable amount of
time, but also should not be so close that it results in the detector having a
high dead time. The spectra used in this analysis are shown in Figure \label{fig:spectra}.

\begin{figure}[H]
\label{fig:spectra}
\begin{center}
\includegraphics[width=.6\linewidth]{../fig/spectra.png}
\caption{Spectra used in calibration}
\end{center}
\end{figure}

The lines of interest, given in Table
\ref{tab:src} \cite{lund}, were primarily chosen based on their
branching ratios, as those with the highest branching ratios tend to be
the most visible within a spectrum.

\begin{table}[H]
  \begin{center}
    \begin{tabular}{l|r}
      \textbf{Source} & \textbf{Energy (keV)}\\
      \hline
      $^{241}$Am    &  59.541    \\
      $^{133}$Ba    &  80.997    \\
                    &  356.017   \\
      $^{60}$Co     &  1173.237  \\
                    &  1332.501  \\
      $^{137}$Cs    &  661.657   \\
      $^{152}$Eu    &  121.781   \\
                    &  1408.006  \\
    \end{tabular}
    \caption{Gamma-ray lines used in the calibration}
    \label{tab:src}
  \end{center}
\end{table}

For each gamma-ray energy of interest, the line was found by inspecting the spectrum of
the associated source and then fit with a Gaussian function in the form

\begin{equation}
G(x; A, \mu, \sigma) = A\exp\bigg(-\frac{(x-\mu)^2}{2\sigma^2}\bigg)
\end{equation}

The relevant parameter for an energy calibration is the centroid $\mu$ of the fit,
though it is worth noting that the other parameters also contain valuable information,
for example, in determining the full-width at half-max (FWHM) energy resolution.

After determining the centroid $C_i = \mu_i$ for each energy of interest, a
regression is performed between $C_i$ and the associated energy $E_i$. While there
may exist slight nonlinearities in the relationship between the MCA channel and
energy, often times, a linear fit between $C_i$ and $E_i$ is sufficient.
In particular, we wish to find coefficients
$\vec{a} = \begin{bmatrix}a_0, a_1 \end{bmatrix}$ such that

\begin{equation}
    \begin{bmatrix}
        E_0 \\ E_1 \\ \vdots \\ E_n
    \end{bmatrix} =
    \begin{bmatrix}
        1 & C_0 \\
        1 & C_1 \\
        \vdots & \vdots \\
        1 & C_n \\
    \end{bmatrix}
    \begin{bmatrix}
        a_0 \\ a_1
    \end{bmatrix}
\end{equation}
where $n$ is the number of gamma-ray lines used. This equation can then be
inverted (e.g., using least squares) to yield a linear model.
