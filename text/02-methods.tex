\subsection{Spectroscopic Gamma-ray Data}
The theory behind signal generation is HPGe detectors is described in detail in \cite{gilmore_2011}.
The spectroscopic information of a particular measured gamma-ray event $i$ is encoded as a voltage $V_i$.
When this signal is passed to a multi-channel analyzer (MCA), the MCA increments the number of counts in bin $k$, where $k$ is such that $V_k \le V_i \le V_k + \Delta V$, where $\Delta V$ is the width of the histogram bin.
All events recorded within a period $\Delta t$, referred to as the integration time, are aggregated to form a gamma-ray spectrum.
The result of the measurement is a histogram $\frac{dN}{dE}$ with total number of counts $n_k$ in the the $k$th bin.

\subsection{Spectroscopic Calibration}
To perform a calibration, we first choose a gamma-ray source with well-characterized energies.
The source should be of high enough activity such that the resulting peaks can be easily determined.
As a result, common lab sources typically have half-lives on the order of years to tens of years to remain useful for an extended period of time.

When performing the calibration, the source is placed in front of a gamma-ray detector and recorded for some integration time $\Delta t$.
If the calibration is being performed using multiple sources, the source location should be the same for each source, and should be close enough to yield a high number of counts in a reasonable amount of time, but also should not be so close that it results in the detector having a high dead time.
After recording the spectrum and identifying a particular source peak, we can then find the centroid of this peak within the histogram.
For an accurate determination of the peak centroid, we can find the best fit of a unimodal function to the peak, and extract the centroid information.
A common choice for a function to fit to the peak is a Gaussian function of the form
\begin{equation}
f(x; A, \mu, \sigma) = A\exp\bigg(-\frac{(x-\mu)^2}{2\sigma^2}\bigg)
\end{equation}
where $A$ is the peak amplitude, $\mu$ is the centroid, and $\sigma^2$ is the variance.

After determining the peak centroid $C_i = \mu_i$ for each energy of interest, a regression is performed between $C_i$ and the associated energy $E_i$.
While there may exist slight nonlinearities in the relationship between the MCA channel and energy, often times, a linear fit (i.e., first-order polynomial) between $C_i$ and $E_i$ is sufficient.
In the case of a first-order polynomial fit, we wish to find coefficients
$\mathbf{a} = \begin{bmatrix}a_0 & a_1 \end{bmatrix}^T$ such that

\begin{equation}
    \begin{bmatrix}
        E_0 \\ E_1 \\ \vdots \\ E_n
    \end{bmatrix} =
    \begin{bmatrix}
        1 & C_0 \\
        1 & C_1 \\
        \vdots & \vdots \\
        1 & C_n \\
    \end{bmatrix}
    \begin{bmatrix}
        a_0 \\ a_1
    \end{bmatrix}
\end{equation}
where $n$ is the number of gamma-ray lines used. This equation can then be inverted (e.g., using linear regression) to yield weights $\mathbf{a}$ that form a linear model.
This linear transformation can then be used to find the gamma-ray energy corresponding to an arbitrary bin in the spectrum.

\subsection{Measurements}
Measurements of various gamma-ray sources using a coaxial HPGe were performed and provided by by Dr. Ross Barnowski.
The measurements were performed using a 13-bit resolution MCA, yielding 8192-bin spectra.
The procedure for taking the measurements involved placing a source at a specific location and recording counts for a period of time, and repeating for each source.
The spectra used in this analysis are shown in \Fref{fig:spectra}.

\begin{figure}
\centering
\includegraphics[width=3.5in]{spectra.eps}
\caption{Spectra of different sources captured by the HPGe detector used in the calibration procedure. The energies for a number of the dominant lines in these spectra are given in Table \ref{tab:src}.}
\label{fig:spectra}
\end{figure}
The sources used in this analysis, as well as their dominant gamma-ray lines, are given in Table \ref{tab:src}.
Note that the sources used range from tens of keV to slightly over 1.4 MeV, and as a result, may not accurately describe behavior at higher energies in the range we are considering.

\begin{table}
\renewcommand{\arraystretch}{1.3}
\caption{Gamma-ray lines used in the calibration \cite{lund}}
\label{tab:src}
\centering
\begin{tabular}{l|r}
\hline
\bfseries Source & \bfseries Energy (keV)\\
\hline\hline
      $^{241}$Am    &  59.541    \\
      $^{133}$Ba    &  80.997    \\
                    &  356.017   \\
      $^{60}$Co     &  1173.237  \\
                    &  1332.501  \\
      $^{137}$Cs    &  661.657   \\
      $^{152}$Eu    &  121.781   \\
                    &  1408.006  \\
\hline
\end{tabular}
\end{table}
