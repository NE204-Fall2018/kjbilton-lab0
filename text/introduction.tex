One of the key values of radiation detection and measurement, and in particular,
gamma-ray spectroscopy, is the ability to ability to provide information regarding
the radiation field at a particular point. For terrestrial applications, we
often wish to relate this measured radiation field to the material composition of
our local environment. Our ability to relate gamma-ray measurements to the material
makeup of our surroundings ultimately comes from the discrete nuclear transitions
that occur within radionuclides. In particular, we rely on the energy signatures
provided by radionuclides. This, of course, ultimately means that we need a
method to faithfully relate the output signals of our gamma-ray spectrometers
to energies which have significance to us.


Spectroscopic calibrations are a routine procedure which must be performed on
gamma-ray detectors to confidently map electronic signals to operator-interpretable
measurements. In particular, by relating known inputs (i.e., specific, controlled
gamma-ray sources) to their respective outputs of our detection systems (i.e.,
voltages, or bins in a multichannel analyzer (MCA)), we can quantitatively determine the
mapping from electrical signal to human-interpretable measurements, yielding
the ability to understand new, unknown sources.
