One of the key values of radiation detection and measurement, and in particular,
gamma-ray spectroscopy, is the ability to ability to provide information regarding
the radiation field at a particular point in space and time. For terrestrial
applications, we often wish to relate this measured radiation field to the material
composition of our local environment. Our ability to relate gamma-ray measurements
to our surroundings ultimately comes from the discrete state transitions that occur within
nulcei, and in particular, it relies on the energy signatures provided by transitions.
Our ability to identify radionuclides of interest then requires a method to faithfully relate the
output signals of our gamma-ray spectrometers to associated gamma-ray lines.


Spectroscopic calibrations are a routine procedure performed on gamma-ray detectors
to confidently map electronic signals to operator-interpretable
measurements. In particular, by relating known inputs (i.e., specific gamma-ray sources)
to their respective outputs of our detection systems (i.e., voltages, or bins in a multichannel analyzer (MCA)),
we can quantitatively determine the
mapping from electrical signal to human-interpretable measurements, yielding
the ability to understand new, unknown sources.
