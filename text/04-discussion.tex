The resulting fit from our linear regression can then be used for mapping
MCA channels to gamma-ray energy. For routine lab meaurements and estimations, experts often time only use two energy measurements from a single source to determine a linear fit \cite{gilmore_2011}.
While a linear fit is generally sufficient, some specific applications may require nonlinear fitting.
This can easily be accomplished by including higher-orders of the centroid $C_i$
\begin{equation}
    \begin{bmatrix}
        E_0 \\ E_1 \\ \vdots \\ E_n
    \end{bmatrix} =
    \begin{bmatrix}
        1 & C_0  & C_0^2 & \dots &  C_0^m\\
        1 & C_1  & C_1^2 & \dots &  C_1^m\\
        \vdots & \vdots \\
        1 & C_n  & C_n^2 & \dots &  C_n^m\\
    \end{bmatrix}
    \begin{bmatrix}
        a_0 \\ a_1 \\ a_2 \\ \vdots \\ a_m
    \end{bmatrix}
\end{equation}
As before, the weights $\mathbf{a}$ can be found using OLS.

Aside from using a higher-order model, one could also improve the fit used to
determine the peaks. Here, a Gaussian model was assumed, and no linear offset
or background subtraction was performed. To more accurately determine the peak centers,
one could fit a Gaussian superimposed on a linear model. In our case, the
simplified fit provided results sufficient for the purpose of this exercise,
so only a single Gaussian peak was used.
