\IEEEPARstart{O}{ne} of the key values of radiation detection and measurement, and in particular, gamma-ray spectroscopy, is the ability to ability to provide information regarding the radiation field at a particular location.
For terrestrial applications, we often wish to relate a measured radiation field to the material composition of our local environment.
Our ability to relate gamma-ray measurements to our surroundings ultimately comes from the discrete state transitions that occur within nuclei, and in particular, it relies on the energy signatures provided by transitions.
Our ability to identify radionuclides of interest then requires a method to faithfully relate the output signals of our gamma-ray spectrometers to associated gamma-ray lines.

Spectroscopic calibrations are a routine procedure performed on gamma-ray detectors to map electronic signals corresponding to measured gamma-ray events to operator-interpretable units (i.e., gamma-ray energy).
In particular, by relating known inputs (i.e., specific gamma-ray source energies) to their respective outputs of our detection systems (i.e., voltages), we can quantitatively determine the mapping from electrical signal to human-interpretable measurements.
By performing calibrations with known sources, we gain the ability to characterize gamma rays from additional sources.

In this work, we consider calibration procedures for high-purity germanium (HPGe) gamma-ray detectors.
While there are significant differences between HPGe and other common gamma-ray detectors, namely scintillation detectors (e.g., \ce{NaI(Tl)}), the methods described here can be followed as such detectors.

The remainder of this paper is outlined as follows.
\Fref{sec:meth} describes the methods involved in performing calibrations.
\Fref{sec:res} shows results for performing spectroscopic calibrations on measured gamma-ray spectra.
\Fref{sec:disc} closes with a discussion of results.
