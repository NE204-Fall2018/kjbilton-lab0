\documentclass[journal]{IEEEtran}

%==============================================================================
%%% Everything between the "="'s is the preamble.
%%% Define packages and meta data here

% Common packages
\usepackage{amsmath}    % Expanded math
\usepackage{amssymb}    % Expanded math symbols
\usepackage{graphicx}   % For images
\usepackage{float}
\usepackage[plain]{fancyref}
\usepackage[version=3]{mhchem} % For nuclide formatting

% All images/figures will be stored in the images folder.
% Specify that here so pdflatex knows where to look for images.
\graphicspath{{./fig/}}

% Metadata
\title{On the Calibration of Spectroscopic Data from High-purity Germanium Gamma-ray Detectors}
\author{Kyle J. Bilton%
\thanks{Kyle~J.~Bilton is with the Department of Nuclear Engineering at the University of California, Berkeley, Berkeley, CA 94720 USA. (email:~\mbox{kjbilton@berkeley.edu})}}

%==============================================================================

\begin{document}

\maketitle

\begin{abstract}
To provide meaningful, interpretable results, gamma-ray spectrometers must be calibrated.
Here, a procedure is outlined for calibrations of spectroscopic gamma-ray from high-purity germanium (HPGe) detectors.
In particular, ordinary least squares regression is performed between electronic signals resulting from measurements of standard gamma-ray sources and their known energies.
The calibration procedure is demonstrated using five common lab sources.
The resulting relationship between channel number and energy shows to generalize well within the energy range of interest.
The methods presented in this paper are broadly applicable to HPGe detectors, as well as other standard spectroscopic gamma-ray detectors.
\end{abstract}


\section{Introduction}
\label{sec:intro}
\IEEEPARstart{O}{ne} of the key values of radiation detection and measurement, and in particular, gamma-ray spectroscopy, is the ability to ability to provide information regarding the radiation field at a particular location.
For terrestrial applications, we often wish to relate a measured radiation field to the material composition of our local environment.
Our ability to relate gamma-ray measurements to our surroundings ultimately comes from the discrete state transitions that occur within nuclei, and in particular, it relies on the energy signatures provided by transitions.
Our ability to identify radionuclides of interest then requires a method to faithfully relate the output signals of our gamma-ray spectrometers to associated gamma-ray lines.

Spectroscopic calibrations are a routine procedure performed on gamma-ray detectors to map electronic signals corresponding to measured gamma-ray events to operator-interpretable units (i.e., gamma-ray energy).
In particular, by relating known inputs (i.e., specific gamma-ray source energies) to their respective outputs of our detection systems (i.e., voltages), we can quantitatively determine the mapping from electrical signal to human-interpretable measurements.
By performing calibrations with known sources, we gain the ability to characterize gamma rays from additional sources.

In this work, we consider calibration procedures for high-purity germanium (HPGe) gamma-ray detectors.
While there are significant differences between HPGe and other common gamma-ray detectors, namely scintillation detectors (e.g., \ce{NaI(Tl)}), the methods described here can be followed as such detectors.

The remainder of this paper is outlined as follows.
\Fref{sec:meth} describes the methods involved in performing calibrations.
\Fref{sec:res} shows results for performing spectroscopic calibrations on measured gamma-ray spectra.
\Fref{sec:disc} closes with a discussion of results.


\section{Methods}
\label{sec:meth}
\subsection{Spectroscopic Gamma-ray Data}
The theory behind signal generation is HPGe detectors is described in detail in \cite{gilmore_2011}.
The spectroscopic information of a particular measured gamma-ray event $i$ is encoded as a voltage $V_i$.
When this signal is passed to a multi-channel analyzer (MCA), the MCA increments the number of counts in bin $k$, where $k$ is such that $V_k \le V_i \le V_k + \Delta V$, where $\Delta V$ is the width of the histogram bin.
All events recorded within a period $\Delta t$, referred to as the integration time, are aggregated to form a gamma-ray spectrum.
The result of the measurement is a histogram $\frac{dN}{dE}$ with total number of counts $n_k$ in the the $k$th bin.

\subsection{Spectroscopic Calibration}
To perform a calibration, we first choose a gamma-ray source with well-characterized energies.
The source should be of high enough activity such that the resulting peaks can be easily determined.
As a result, common lab sources typically have half-lives on the order of years to tens of years to remain useful for an extended period of time.

When performing the calibration, the source is placed in front of a gamma-ray detector and recorded for some integration time $\Delta t$.
If the calibration is being performed using multiple sources, the source location should be the same for each source, and should be close enough to yield a high number of counts in a reasonable amount of time, but also should not be so close that it results in the detector having a high dead time.
After recording the spectrum and identifying a particular source peak, we can then find the centroid of this peak within the histogram.
For an accurate determination of the peak centroid, we can find the best fit of a unimodal function to the peak, and extract the centroid information.
A common choice for a function to fit to the peak is a Gaussian function of the form
\begin{equation}
f(x; A, \mu, \sigma) = A\exp\bigg(-\frac{(x-\mu)^2}{2\sigma^2}\bigg)
\end{equation}
where $A$ is the peak amplitude, $\mu$ is the centroid, and $\sigma^2$ is the variance.

After determining the peak centroid $C_i = \mu_i$ for each energy of interest, a regression is performed between $C_i$ and the associated energy $E_i$.
While there may exist slight nonlinearities in the relationship between the MCA channel and energy, often times, a linear fit (i.e., first-order polynomial) between $C_i$ and $E_i$ is sufficient.
In the case of a first-order polynomial fit, we wish to find coefficients
$\mathbf{a} = \begin{bmatrix}a_0 & a_1 \end{bmatrix}^T$ such that

\begin{equation}
    \begin{bmatrix}
        E_0 \\ E_1 \\ \vdots \\ E_n
    \end{bmatrix} =
    \begin{bmatrix}
        1 & C_0 \\
        1 & C_1 \\
        \vdots & \vdots \\
        1 & C_n \\
    \end{bmatrix}
    \begin{bmatrix}
        a_0 \\ a_1
    \end{bmatrix}
\end{equation}
where $n$ is the number of gamma-ray lines used. This equation can then be inverted (e.g., using linear regression) to yield weights $\mathbf{a}$ that form a linear model.
This linear transformation can then be used to find the gamma-ray energy corresponding to an arbitrary bin in the spectrum.

\subsection{Measurements}
Measurements of various gamma-ray sources using a coaxial HPGe were performed and provided by by Dr. Ross Barnowski.
The measurements were performed using a 13-bit resolution MCA, yielding 8192-bin spectra.
The procedure for taking the measurements involved placing a source at a specific location and recording counts for a period of time, and repeating for each source.
The spectra used in this analysis are shown in \Fref{fig:spectra}.

\begin{figure}
\centering
\includegraphics[width=3.5in]{spectra.eps}
\caption{Spectra of different sources captured by the HPGe detector used in the calibration procedure. The energies for a number of the dominant lines in these spectra are given in Table \ref{tab:src}.}
\label{fig:spectra}
\end{figure}
The sources used in this analysis, as well as their dominant gamma-ray lines, are given in Table \ref{tab:src}.
Note that the sources used range from tens of keV to slightly over 1.4 MeV, and as a result, may not accurately describe behavior at higher energies in the range we are considering.

\begin{table}
\renewcommand{\arraystretch}{1.3}
\caption{Gamma-ray lines used in the calibration \cite{lund}}
\label{tab:src}
\centering
\begin{tabular}{l|r}
\hline
\bfseries Source & \bfseries Energy (keV)\\
\hline\hline
      $^{241}$Am    &  59.541    \\
      $^{133}$Ba    &  80.997    \\
                    &  356.017   \\
      $^{60}$Co     &  1173.237  \\
                    &  1332.501  \\
      $^{137}$Cs    &  661.657   \\
      $^{152}$Eu    &  121.781   \\
                    &  1408.006  \\
\hline
\end{tabular}
\end{table}


\section{Results}
\label{sec:res}
In the simplest case, we can use two points to determine a linear relationship using ordinary least squares (OLS) regression.
We used a two point fit with \ce{$^{137}$Cs} and \ce{$^{241}$Am} to give the model

\begin{equation}
\label{eq:cal}
E_k = 1.181 + 0.281k
\end{equation}
where $k$ is the bin number.


To evaluate the performance of this model, we estimate at which bins lines from \ce{$^{133}$Ba} are expected to fall in, and compare this to what has been measured.
Table \ref{tab:barium} shows measured and estimated energies for \ce{^{133} Ba}, as well as the percent error.
We see that the two estimated energies are quite close to the actual energies.

\begin{table}
\renewcommand{\arraystretch}{1.3}
\caption{Measured and estimated source energies for \ce{^{133} Ba}}
\label{tab:barium}
\centering
\begin{tabular}{l|c|c|r}
\hline
\bfseries Source & \bfseries Energy (keV) & \bfseries Estimated Energy (keV) & \bfseries Percent Error\\
\hline\hline
      $^{133}$Ba    &  80.997    & 80.684   &  0.1632 \\
                    &  356.017   & 356.110  &  -0.0261 \\
\hline
\end{tabular}
\end{table}

In addition to the two points used here, it is also worth considering using additional data points, including some at higher energies.
OLS was performed on several points to obtain the line below.
The solid line in Figure \ref{fig:fit} shows a fit using all data points, while the dashed line shows the two-point relationship.

\begin{figure}
\centering
\includegraphics[width=3.5in]{calibration_fit.eps}
\caption{Resulting linear fit of gamma-ray energies to channel number. The solid red line shows the fit using all data points, while the dashed blue results from the two-point fit.}
\label{fig:fit}
\end{figure}

The spectra shown in \Fref{fig:spectra} could then be translated into energies using \Fref{eq:cal}.
Calibrated spectra are shown in \Fref{fig:spectra_cal}.
\begin{figure*}[t!]
\centering
\includegraphics[width=6.5in]{calibrated_spectra.eps}
\caption{Energy-calibrated spectra captured by the HPGe used in the calibration procedure. \ce{^{241}Am} (left), \ce{^{133}Ba} and \ce{^{137}Cs} (center), and \ce{^{60}Co} and \ce{^{152}Eu} (right).}
\label{fig:spectra_cal}
\end{figure*}


\section{Discussion}
\label{sec:disc}
The resulting fit from our linear regression can then be used for mapping
MCA channels to gamma-ray energy. For routine lab meaurements and estimations, experts often time only use two energy measurements from a single source to determine a linear fit \cite{gilmore_2011}.
While a linear fit is generally sufficient, some specific applications may require nonlinear fitting.
This can easily be accomplished by including higher-orders of the centroid $C_i$
\begin{equation}
    \begin{bmatrix}
        E_0 \\ E_1 \\ \vdots \\ E_n
    \end{bmatrix} =
    \begin{bmatrix}
        1 & C_0  & C_0^2 & \dots &  C_0^m\\
        1 & C_1  & C_1^2 & \dots &  C_1^m\\
        \vdots & \vdots \\
        1 & C_n  & C_n^2 & \dots &  C_n^m\\
    \end{bmatrix}
    \begin{bmatrix}
        a_0 \\ a_1 \\ a_2 \\ \vdots \\ a_m
    \end{bmatrix}
\end{equation}
As before, the weights $\mathbf{a}$ can be found using OLS.

Aside from using a higher-order model, one could also improve the fit used to
determine the peaks. Here, a Gaussian model was assumed, and no linear offset
or background subtraction was performed. To more accurately determine the peak centers,
one could fit a Gaussian superimposed on a linear model. In our case, the
simplified fit provided results sufficient for the purpose of this exercise,
so only a single Gaussian peak was used.


% Bibliography
\bibliographystyle{IEEEtran}
\bibliography{references}

\end{document}
